\documentclass[a4paper, 10pt]{article}
\usepackage{comment} % enables the use of multi-line comments (\ifx \fi) 
\usepackage{lipsum} %This package just generates Lorem Ipsum filler text. 
\usepackage{fullpage} % changes the margin
\usepackage{epsfig}
\usepackage{listings}

\begin{document}
%Header-Make sure you update this information!!!!
\noindent
\large\textbf{Lab Report:1} \hfill \textbf{Abhishek Srivastava \& Ravdeep Pasricha} \\
\normalsize CS202 Advanced Operating Systems\hfill Student Id: 861307778 \& 86130XXXX \\
Prof. Nael B. Abu-Ghazaleh \hfill \today \\
\hrule

\noindent
\\
\large\textbf{Problem 1}\\
Prob 1 description.\\

\newpage

\noindent
\large\textbf{Problem 2}\\
Lottery \& Stride Scheduler.\\
 
\noindent
\large\textbf{Solution:}\\
Files Modified:
\begin{itemize}
	\item \textbf{defs.h}: Modified for adding system call  \emph{int sticket(int)}.
	\item \textbf{proc.c}:  Added the definition of \emph{int sticket(int)} to assign the ticket value to the process , increase the count of total tickets, assign the stride value and total stride value for a process, and also modified the round robin scheduler by adding Lottery Scheduler and Stride Scheduler. 
	\item \textbf{proc.h}: Modified \emph{proc} data structure to add 4 more parameters \emph{int tcount,	int numticks, int stride,	int tstride} for lottery and stride scheduler implementation.
	\item \textbf{syscall.h}, \textbf{syscall.c}, \textbf{sysproc.c}, \textbf{usys.S}, \textbf{user.h}: Modified to add System cal to set ticket, stride value for process.
	\item \textbf{Makefile}:  Added test files \emph{prog1.c, prog2.c, prog3.c, lprog1.c, lprog2.c and lprog3.c}.\\
\end{itemize}
Code Added:

\begin{lstlisting}
//System call to set values for the process.
int sticket(int tnum)
{
    //Set the number of tickets to value passed
    proc->tcount = tnum;
    //counter for total number of tickets
    totticket += tnum;
    //calculate the stride value for process
    proc->stride = totstride/tnum;
    proc->tstride = totstride/tnum;
    //For debug purposes
    //cprintf(``ticket value set to : %d\n'', proc->tcount);
    //cprintf(``stride value set to : %d\n'', proc->stride);
}

/*------------------------------------------------------------------*/

// Per-process state
struct proc {
    uint sz;                 // Size of process memory (bytes)
    pde_t* pgdir;            // Page table
    char *kstack;            // Bottom of kernel stack for this process
    enum procstate state;    // Process state
    int pid;                 // Process ID
    struct proc *parent;     // Parent process
    struct trapframe *tf;    // Trap frame for current syscall
    struct context *context; // swtch() here to run process
    void *chan;              // If non-zero, sleeping on chan
    int killed;              // If non-zero, have been killed
    struct file *ofile[NOFILE];  // Open files
    struct inode *cwd;       // Current directory
    char name[16];           // Process name (debugging)
    int tcount;              // CS202 : Ticket Count
    int numticks;            // CS202 : Number of times it is scheduled
    int stride;              // CS202 : Stride value
    int tstride;             // CS202 : Total stride value
};

/*------------------------------------------------------------------*/

//Stride Scheduler Implementation
void stride_scheduler(void)
{
    struct proc *p;

    for(;;){
        // Enable interrupts on this processor.
        sti();

        //Min stride value holder, highest initial value
        int min_stride = totstride;
        //To store the pid of process with minimum stride value
        int stride_pid = 0;

        // Loop over process table looking for process to run.
        acquire(&ptable.lock);

        //Loop over all the process to find out process with
        //minimum stride value
        for(p = ptable.proc; p < &ptable.proc[NPROC]; p++){
        if(p->state != RUNNABLE)
            continue;

        //Check if process has the less stride value
        if(p->tstride < min_stride)
        {
            min_stride = p->tstride;
            //store pid value
            stride_pid = p->pid;
        }
    }
    //Choose the process with pid of minimum stride value
    for(p = ptable.proc; p < &ptable.proc[NPROC]; p++){
        if(p->state != RUNNABLE)
            continue;

        //check if process has the pid off minimum stride value
        if(p->pid != stride_pid)
            continue;

        // Switch to chosen process.  It is the process's job
        // to release ptable.lock and then reacquire it
        // before jumping back to us.

        //Increase the count of scheduled number
        p->numticks +=1;
        //Increasing the total stride value
        p->tstride += p->stride; 
        proc = p;
        switchuvm(p);
        p->state = RUNNING;
        swtch(&cpu->stride_scheduler, p->context);
        switchkvm();

        //Process is done running for now.
        //It should have changed its p->state before coming back.
        proc = 0;
        }
        release(&ptable.lock);
    }
}

/*------------------------------------------------------------------*/

//Lottery Scheduler Implementation
void lottery_scheduler(void)
{
    struct proc *p;

    for(;;){
        // Enable interrupts on this processor.
        sti();

        //generate random ticket
        int winner = rand()%(totticket+1)
        //To hold total number of tickets traverssed
        int ticketsum = 0;

        // Loop over process table looking for process to run.
        acquire(&ptable.lock);
        for(p = ptable.proc; p < &ptable.proc[NPROC]; p++){
            if(p->state != RUNNABLE)
                continue;

            //condition to choose the process which are runnable
            ticketsum += p->tcount;
            if(ticketsum < winner)
                continue;

            // Switch to chosen process.  It is the process's job
            // to release ptable.lock and then reacquire it
            // before jumping back to us.

            //Increase the count of scheduled number
            p->numticks +=1;
            proc = p;
            switchuvm(p);
            p->state = RUNNING;
            swtch(&cpu->lottery_scheduler, p->context);
            switchkvm();

            // Process is done running for now.
            // It should have changed its p->state before coming back.
            proc = 0;
        }
        release(&ptable.lock);
    }
}

/*------------------------------------------------------------------*/
\end{lstlisting}
\end{document}
