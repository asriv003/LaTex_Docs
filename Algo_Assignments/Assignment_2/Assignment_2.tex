% homework template
\documentclass[a4paper,11pt]{article}

%package declarations:
%geometry:set the geometry of page
%ragged2e: left/right justify
%fancyhdr: header/footers
\usepackage{geometry}
\usepackage{ragged2e}
\usepackage{fancyhdr}
\usepackage{amsmath,amssymb,amsthm,graphicx}
\usepackage{algorithm}
\usepackage{algpseudocode}
\usepackage{wrapfig}

%redefine maketitle
%http://tex.stackexchange.com/questions/85343/left-align-abstract-title-and-authors
\renewcommand{\maketitle}{%
	
	\Large
 	\textbf{CS 218, Spring 2017}
 	\hfill
 	Homework 2
 	\par
 	
	\Large
	Abhishek Kumar SRIVASTAVA
	\hfill
	\normalsize
	\today
 	\par
 	Student ID: 861307778
 	\par
 	
 	%plagiarism statement
 	\begin{center}

 	\vspace{.2in}
 	
 	\textit{I certify that this submission represents my own original work }
 	\par
	\vspace{.2in}
	\makebox[3.0in]{\hrulefill}
	\par

 	\end{center}
 	
 	\hrulefill
 	\par \vspace{2ex}
 	}


%since using the assignment class, set the geometry
\geometry{total={210mm,297mm},
	left=25mm,right=25mm,%
	bindingoffset=0mm, top=20mm,bottom=20mm}

%set headers and footers
\pagestyle{fancy}
\fancyhf{}
\fancyhead[LE,RO]{\textbf{CS 218:} Homework 2}
\fancyhead[RE,LO]{Abhishek Kumar Srivastava}
\fancyfoot[CE,CO]{\leftmark}
\fancyfoot[LE,RO]{\thepage}

%some custom commands, taken from stanford template
\newtheoremstyle{quest}{\topsep}{\topsep}{}{}{\bfseries}{}{ }{\thmname{#1}\thmnote{ #3}.}
\theoremstyle{quest}
\newtheorem*{definition}{Definition}
\newtheorem*{theorem}{Theorem}
\newtheorem*{question}{Question}
\newtheorem*{exercise}{Exercise}
\newtheorem*{challengeproblem}{Challenge Problem}
\newenvironment{solution}[2][Solution]{\begin{trivlist}
		\item[\hskip \labelsep {\bfseries #1}\hskip \labelsep {\bfseries #2.}]}{\end{trivlist}}

\begin{document}\thispagestyle{empty}
	
\maketitle

\begin{solution}1	
	Given that 
	\begin{align*}
	T(n)&=\begin{cases}
	2 & \text{$n = 2$}\\
	4 \cdot T({n}^{1/2}) + {\log}^2(n) & \text{$n > 2$} \\
	\end{cases}\\
	\text{Consider when $n > 2$}\\
	T(n)&= 4 \cdot T({n}^{1/2}) + {\log}^2(n) \\
	\end{align*}
	Master theorem cannot be applied on this equation. We need to apply transformation to convert into Master theorem equation.
	
	We can substitute $n = 2^m$ in the equation. Which will give us
	\begin{align*}
	T(2^m)&= 4 \cdot T(2^{m/2}) + \log^2(2^m) \\
	&=  4 \cdot T(2^{m/2}) + m^2 \\
	\end{align*}
	Substituting again in this equation by $T(2^m) = S(m)$. We get,
	\begin{align*}
	S(m) = 4 \cdot S(m/2) + m^2
	\end{align*}
	Now we can apply Master theorem on the above equation.\\

	\textbf{Case 1:} $f(n)$ $\epsilon$ $O(n^{\log_b{a} - \varepsilon})$ where $\varepsilon > 0$.\\
	
	$m^2$ $\epsilon$ $O(m^{\log_2{4} - \varepsilon})$ = $O(m^{2 - \varepsilon})$\\
	
	For any $\varepsilon > 0$ this is not possible. So case 1 fails.\\
	
	\textbf{Case 2:} $f(n)$ $\epsilon$ $\Theta(n^{\log_b{a}}\cdot \log^k n)$ where $k \ge 0$.\\
	
	$m^2$ $\epsilon$ $\Theta(m^{\log_2{4}}\cdot \log^k m)$ = $\Theta(m^{2}\cdot \log^k m)$\\
	
	$k = 0$ satisfy the equation. So case 2 holds. Thats why,
	
	\begin{align*}
	S(m) &= \Theta(m^{\log_2{4}}\cdot \log^{k+1}m)\\
	&= \Theta(m^{2}\cdot \log{m})
	\end{align*}
	By substituting back $m = \log{n}$ we get,
	\begin{align*}
	\boxed{T(n) = \Theta(\log^2{n}\cdot \log{\log{n}})}
	\end{align*}
\end{solution}

\newpage
\begin{solution}2
	Given $A[1,\dots,n]$ as a fixed array of distinct integers in which we have to find the position for the elements in array $X[1,\dots,k]$.\\
	
	\textbf{Naive Solution:}\\
	
	We can search the positions of each element in $X[1,\dots,k]$ by doing linear search in array $A[1,\dots,n]$ for that element. Which will the complexity of
	\begin{equation*}
	T(n) = O(n\cdot k)
	\end{equation*}
	But this is not the lower bound. \\
	
	\textbf{Binary Decision Tree Solution:}\\
	
	We can do better than the naive solution to get the lower bound by using binary decision tree model. We first sort the array $A[1,\dots,n]$ and then do a binary search for each element in the array $X[1,\dots,k]$.
	
	Total cost of doing search this way will be cost of sorting the array and searching each element using binary search. Therefore,
	\begin{equation*}
	T(n) = n\cdot \log{n} + k \cdot log{n}
	\end{equation*}
	($n \cdot \log{n}$) since it is the lower bound of sorting $A[1,\dots,n]$ using comparison model.\\
	($k \cdot \log{n}$) since $X[1,\dots,k]$ contains $k$ elements and doing binary search on the sorted array $A[1,\dots,n]$ have the lower bound of $\log{n}$.
	\begin{center}
		$\therefore T(n) = O((n+k)\log{n})$ \\
	\end{center}
	This gives the lower bound on the time complexity, as a function of $n$ and $k$, using the binary decision tree model for searching and sorting.\\
	
	Pseudocode for described algorithm is as following: 
	\begin{algorithm}[h]
		%\caption{}
		\begin{algorithmic}
			\State{Array $A[1,\dots,n]$} \Comment{Fixed Array}
			\State{Array $X[1,\dots,k]$} \Comment{Array to be searched}
			\State{Array $I[1,\dots,k]$} \Comment{Array to store position}\\
			
			\Function{FIND\_POSITION}{$A, X, I$}\\
			\State{SORT($A$)} \Comment{Sort the array $A$ in increasing order}\\
			\For{$i$ less than length($X$)}
			\State{$I[i] =$ BINARY\_SEARCH($A,X[i]$)} \Comment{Search $X[i]$ in $A$ and return its position}
			\EndFor\\
			\EndFunction
		\end{algorithmic}
	\end{algorithm}
\end{solution} 

\newpage
\begin{solution}3  Implement a queue using two stacks $S_1$ and $S_2$.\\
	
	\textbf{Part 1}:
	
	Queue is data structure which follows \textbf{FIFO}(First In First Out) property. On the other hand Stack is data structure which follows \textbf{LIFO}(Last In First Out) property. Therefore to implement Queue using Stack we need two of them. We always treat one stack for en-queuing data and other one for de-queuing. 
	
	While \textsc{Enqueue}$(x)$ operation we push the element $x$ in one stack for example $S1$. For \textsc{Dequeue}$(x)$ operation we first check that other stack($S2$) is empty or not. If it is not empty we pop the top element from $S2$ stack. If it is empty we then transfer all the elements from $S1$ to $S2$ and then perform the pop operation.
	By using this method we can achieve \textbf{FIFO} property. Psuedocode for the above described method is following:
	
	\begin{algorithm}[h]
		%\caption{}
		\begin{algorithmic}
			\State{Stack $S1$} \Comment{Input stack}
			\State{Stack $S2$} \Comment{Output stack}\\
			
			\Function{ENQUEUE}{$x$}
			\State{$S1 \cdot push(x)$} \Comment{Pushing in the stack $S1$}
			\EndFunction \\
			
			\Function{DEQUEUE}{}()
			\If {$S2.empty() = true$} \Comment{Check if $S2$ is empty}
			\State{TRANSFER\_ELEM\_S1\_TO\_S2()} \Comment{Transfer all elements from $S1$ to $S2$}
			\EndIf
			\State{return $S2.pop()$}\Comment{Pop the element from $S2$}
			\EndFunction\\
			
			\Function{TRANSFER\_ELEM\_S1\_TO\_S2}{}()
			\While{$S1.empty() = false$}\Comment{While $S1$ is not empty}
			\State{$S2.push(S1.pop())$}\Comment{Pop elements from $S1$ and push in $S2$}
			\EndWhile
			\EndFunction
		\end{algorithmic}
	\end{algorithm}
	
	\textbf{Part 2}: Accounting Method
	
	Since we are using two stack we will have 4 operation to perform. push and pop for both $S1$ and $S2$.
	\begin{itemize}
		\item \textbf{Proposed charging Scheme} :
		\begin{itemize}
			\item For \textsc{Enqueue}$(x)$ operation we will charge \$4
			\begin{itemize}
				\item \$1 pays for the pushing the element in $S1$. 
				\item \$3 is deposited(credit invariant) on the element to pay \$1 when popping the element from $S1$ later.
				\item \$1 pays for pushing the element in $S2$ from remaining \$2.
				\item \$1 pays for popping the element from $S2$ from the last remaining \$1.
			\end{itemize}
			\item For \textsc{Dequeue}$()$ operation we will charge \$0
		\end{itemize}
	\end{itemize}
	
	For any sequence of $n$ For \textsc{Enqueue}$(x)$ and $n$ \textsc{Dequeue}$()$ operations the cost will be
	\begin{center}
	$T(n) = (4 + \dots \text{n times} \dots +4) + (0 + \dots \text{n times} \dots + 0) = 4n$ \\
	\end{center}
	$\therefore$ Each operation costs
	\begin{center}
		\boxed{$T($n$) = 4$n$/2$n$ = 2 = O(1)$}
	\end{center}
	
	 
	\end{solution}

\end{document}