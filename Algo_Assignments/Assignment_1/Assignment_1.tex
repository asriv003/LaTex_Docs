% homework template
\documentclass[a4paper,11pt]{article}

%package declarations:
%geometry:set the geometry of page
%ragged2e: left/right justify
%fancyhdr: header/footers
\usepackage{geometry}
\usepackage{ragged2e}
\usepackage{fancyhdr}
\usepackage{amsmath,amssymb,amsthm,graphicx}
\usepackage{algorithm}
\usepackage{algpseudocode}
\usepackage{wrapfig}

%redefine maketitle
%http://tex.stackexchange.com/questions/85343/left-align-abstract-title-and-authors
\renewcommand{\maketitle}{%
	
	\Large
 	\textbf{CS 218, Spring 2017}
 	\hfill
 	Homework 1
 	\par
 	
	\Large
	Abhishek Kumar SRIVASTAVA
	\hfill
	\normalsize
	\today
 	\par
 	Student ID: 861307778
 	\par
 	
 	%plagiarism statement
 	\begin{center}

 	\vspace{.2in}
 	
 	\textit{I certify that this submission represents my own original work }
 	\par
	\vspace{.2in}
	\makebox[2.0in]{\hrulefill}
	\par

 	\end{center}
 	
 	\hrulefill
 	\par \vspace{2ex}
 	}


%since using the assignment class, set the geometry
\geometry{total={210mm,297mm},
	left=25mm,right=25mm,%
	bindingoffset=0mm, top=20mm,bottom=20mm}

%set headers and footers
\pagestyle{fancy}
\fancyhf{}
\fancyhead[LE,RO]{\textbf{CS 218:} Homework 1}
\fancyhead[RE,LO]{Abhishek Kumar Srivastava}
\fancyfoot[CE,CO]{\leftmark}
\fancyfoot[LE,RO]{\thepage}

%some custom commands, taken from stanford template
\newtheoremstyle{quest}{\topsep}{\topsep}{}{}{\bfseries}{}{ }{\thmname{#1}\thmnote{ #3}.}
\theoremstyle{quest}
\newtheorem*{definition}{Definition}
\newtheorem*{theorem}{Theorem}
\newtheorem*{question}{Question}
\newtheorem*{exercise}{Exercise}
\newtheorem*{challengeproblem}{Challenge Problem}
\newenvironment{solution}[2][Solution]{\begin{trivlist}
		\item[\hskip \labelsep {\bfseries #1}\hskip \labelsep {\bfseries #2.}]}{\end{trivlist}}

\begin{document}\thispagestyle{empty}
	
\maketitle

\begin{solution}1	
	\textbf{\textit{Step 1:}} Initially $i = 1$. In the \textbf{Loop 1}(while $i \leq n$ do), the value of $i$ increases by the power of 2 till it reaches $n$. Since it is given that we can assume $n$ is a power of 2, then let $m$ be a number such that $2^{m}=n$. %By logarithm calculation, $m=\log_2(n)$.
	
	\textbf{\textit{Step 2:}} For each iteration of \textbf{Loop 1}, in \textbf{Loop 2}(for $j$ = $i$ to $2i-1$) the value of $j$ goes from $i$ to $2i-1$, so this loop runs $2i-1-i+1 = i$ times.
	
	\textbf{\textit{Step 3:}} Therefore, cost of \textbf{Loop 2} is given by the sum of $O(1)$ i.e Constant time print operations, since no other operation is inside the loop.
	Therefore, 
	\begin{align*}
	C_2=\sum_{j=i}^{2i-1} O(1)
	\end{align*}
	
	\text {Since it runs $i$ number of times, it is sum of constant values $i$ number of times.}
	\begin{align*}
	&= O(1) + O(1) + . . . + O(1) \\
	&= i
	\end{align*}
	
	\textbf{\textit{Step 4:}} So the cost of \textbf{Loop 1}  is given by the sum of \textbf{Loop 2} cost($C_2$) for value $i$ takes.
	Therefore, 
	\begin{align*}
	C_1 =\sum_{i=1}^{n} C_2 &=\sum_{i=1}^{n} i \\
	&= 1 + 2 + 4 + ... n 
	\end{align*}
	
	\text{Since $i$ increase by the power of 2 and we have assumed that $n = 2^m$.}

	\begin{align*}
	&= 1 + 2 + 4 + ... 2^m  \\ 
	&= \frac{2^{m+1} - 1}{2-1}\\
	&= {2^{m+1} - 1} = 2.2^m -1 && \text{(Substituting $n=2^m$)}\\
	&= 2.n -1 \leq 2.n
	\end{align*}

	\textbf{\textit{Step 6:}} Therefore, the total cost is cost of \textbf{Loop 1}($C_1$) which runs in less than $2n$ times for a given $n$. Let, $T(n) = n$. To prove the tight bounds, there must exist $c_1$ and $c_2$ such that $c_1 \times g(n) \leq T(n) \leq c_2 \times g(n)$.
	\begin{align*}
	n \leq n < 2n	
	\end{align*}
	Therefore, $c_1=1$, $c_2=2$ and $g(n)= n$.
	Hence, $T(n) \in \Theta(n)$. 
\end{solution}

\newpage
\begin{solution}2
	Since the wall is stretched infinitely in the both direction i have assumed the starting location as 0. From the starting point we will go $2^{i}$ steps first in the right side, come back to starting location and then we will go $2^{i}$ steps in the left side and return back to starting location and increase the value of $i$ by 1 where $i = 0,1...m$ which is nothing but doubling the steps to be taken in the next iteration for either side, we will continue until the door is found. One more assumption i am taking is while going from starting location to the $2^{i}th$ location we will check for the doors at each location but skipping the locations which are already checked that is doors between $0$ to $2^{i-1}$. Psuedo Code to solve this problem is given on page 3.\\
	
	\textbf{Part 1.}\\
	
	The worst case will be when the door is on the left side and let us assume its value $n = 2^{m}+d$, where $ 1 \leq d \leq 2^{m}$ 
	Since we are traversing in both direction the total cost would be:
	\begin{equation*}
	T(n) = 2 \cdot (2\cdot( 1+2+4...2^{m})) + 2\cdot 2^{m+1} + 2^{m}+d
	\end{equation*}
	Inner bracket cost is because we are going to $2^{i}th$ location and coming back and it is multiplied by 2 because we are doing it for both directions i.e Right and Left. $2\cdot 2^{m+1}$ because we are going till $2^{m+1}$ times in the right direction and then coming back to the starting location. $2^{m}+d$ is added because we are going till $2^{m}$ location in the left direction and next $d$ steps to find the door location. Therefore, 
	\begin{equation*}
	T(n) = 4 \cdot ( 1+2+4...2^{m}) + 2\cdot 2^{m+1} + 2^{m}+d
	\end{equation*}
	Which is equivalent to the following equation.
	\begin{equation*}
	T(n) = 4 \cdot (2^{m+1} - 1) + 4\cdot 2^{m} + 2^{m}+d
	\end{equation*}
	After simplifying in $2^{m}$ terms.
	\begin{equation*}
	T(n) = 8 \cdot 2^{m} - 4 + 4\cdot 2^{m} + 2^{m}+d
	\end{equation*}
	\begin{equation*}
	T(n) = 13 \cdot 2^{m} + d -4
	\end{equation*}
	And since $n = 2^{m} + d$,$ 2^{m} = n-d$. By replacing it we can get.
	\begin{equation*}
	T(n) = 13 \cdot (n-d) + d - 4
	\end{equation*}
	\begin{equation*}
	T(n) = 13 \cdot n - 12\cdot d - 4
	\end{equation*}
	\begin{equation*}
	T(n) \leq 13 \cdot n
	\end{equation*}
	$\therefore T(n) = O(n)$\\
	
	\textbf{Part 2.}\\
	
	Since we already got $T(n)$ expression from the Part 1 which is 
		\begin{equation*}
		T(n) = 13 \cdot n - 12\cdot d - 4
		\end{equation*}
	The worst case will be when $d = 1$.Which will the equation as,
	\begin{equation*}
	T(n) = 13 \cdot n - 16
	\end{equation*}
	 We can see the the equation that coefficient of $n$ is 13. which is the constant multiple in the worst case.
	\begin{algorithm}
		\begin{algorithmic}[h]
			\Procedure{FIND\_DOOR\_LOCATION}{$A$}
			
			\If {door at $A[0] = true$} \Comment{Base case}
			\State $door\_location \gets 0$
			\State return $door\_location$
			\EndIf
			
			\State $pos \gets 0$ \Comment{starting position}
			\State $is\_door\_found \gets false$ \Comment{True if door found}
			\State $door\_location \gets 0$
			\State $steps\_to\_take \gets 1$ \Comment{How much step to take}
			
			\While {$is\_door\_found \neq true$} 
			\State Traverse in $Right$ Direction
			\While{$pos \leq steps\_to\_take/2$} \Comment{Skipping previously checked doors} 
			\State $pos++ $ 
			\EndWhile
			
			\While{$pos \leq steps\_to\_take$} \Comment{Check $Right$ Direction}
			\State $pos++$
			
			\If {door at $A[pos] = true$} \Comment{Check Each location}
			\State $door\_location \gets pos$
			\State $is\_door\_found \gets true$
			\State break
			\EndIf
			 
			\EndWhile
			
			\If {$is\_door\_found \neq true$} \Comment{If door not found in $Right$ Side}
			\While{$pos \neq 0$} \Comment{Traversing back to starting position} 
			\State $pos-- $ 
			\EndWhile
			
			\State Traverse $Left$ Direction
			
			\While{$pos \leq steps\_to\_take/2$} \Comment{Skipping previously checked doors} 
			\State $pos++ $ 
			\EndWhile
			\While{$pos \leq steps\_to\_take$} \Comment{Check $Left$ Direction}
			\State $pos++$
			
			\If {door at $A[pos] = true$} \Comment{Check Each location}
			\State $door\_location \gets pos$
			\State $is\_door\_found \gets true$
			\State break
			\EndIf
			
			\EndWhile
			\EndIf
			\If {$is\_door\_found \neq true$} \Comment{If door also not found $Left$ Side}
			\While{$pos \neq 0$} \Comment{Traversing back to starting position} 
			\State $pos-- $ 
			\EndWhile
			
			\State $steps\_to\_take \gets 2*steps\_to\_take$ \Comment{Increasing steps to be taken by 2}
			\EndIf
			\EndWhile
			\State return $door\_location$
			\EndProcedure
		\end{algorithmic}
	\end{algorithm}

\end{solution} 

\newpage
\begin{solution}3
	\begin{enumerate}
		\item Iterative substitutions \\
		Given that 
		\begin{align*}
		T(n)&=\begin{cases}
		1 & \text{$n = 1$}\\
		4 \cdot T(\frac{n}{2}) + 3 & \text{$n > 1$} \\
		\end{cases}\\
		&= 4 \cdot (4 \cdot T(\frac{n}{4})+ 3) + 3 \\
		&= 16 \cdot T(\frac{n}{4}) + 4 \cdot 3 + 3 \\
		&= 16 \cdot (4 \cdot T(\frac{n}{8})+ 3) + 4 \cdot 3 + 3 \\
		&= 64 \cdot T(\frac{n}{8}) + 16 \cdot 3 + 4 \cdot 3 + 3 \\
		\cdot \\
		\cdot \\
		\cdot \\
		&= 4^{i}.T(\frac{n}{2^{i}}) + 4^{i-1} \cdot 3 + 4^{i-2} \cdot 3 + ...+ 4^{0} \cdot 3 \\
		&= 4^{i}\cdot T(\frac{n}{2^{i}}) + 3\cdot(4^{i-1} + 4^{i-2} + ...+ 4^{0}) \\
		\text{Geometric series for ratio r $=$ 4} \\
		&= 4^{i}\cdot T(\frac{n}{2^{i}}) + 3\cdot(\frac{4^{i} - 1}{4-1} )\\
		&= 4^{i}\cdot T(\frac{n}{2^{i}}) + {4^{i} - 1}\\
		\text {$i = \log_2(n)$ will give us the base case},\\
		&= 4^{\log_2(n)}\cdot T(1) + 4^{\log_2(n)} -1 && \text{(since $i = \log_2(n)$)}\\
		&= 2 \cdot 4^{\log_2(n)} -1\\
		&= 2\cdot 2^{\log_2(n^{2})} -1 && \text{(log manipulation)}\\
		\therefore	T(n)&= 2 \cdot n^{2} - 1 \\	
		\end{align*}
		
		\item
		\textbf{Proof by induction}\\
		Given that 
		\begin{align*}
		T(n)&=\begin{cases}
		1 & \text{$n = 1$}\\
		4 \cdot T(\frac{n}{2}) + 3 & \text{$n > 1$} \\
		\end{cases}
		\end{align*}
		
		\text{\textit{From Iterative Substitution:}} $T(n) = 2 \cdot n^{2} - 1$\\
		
		\textbf{Proof:} Base case: $n = 2$\\
		Calculate using given recurrence relation.
		\begin{align*}
		& T(2) = 4 \cdot T(\frac{2}{2}) + 3\\
		& T(2) = 4 \cdot 1 + 3 = 7 && \text{Since, T(1) = 1}
		\end{align*}
		Calculate using derived asymptotic solution.
		\begin{align*}
		& T(2) = 2 \cdot n^{2} -1\\
		& T(2) = 2 \cdot 2^{2} - 1 = 7	&& \text{Since, n = 2}	
		\end{align*}
		Since both result are the same hence derived solution holds.		\newpage
		Induction hypothesis: T($\frac{n}{2}$) = $2 \cdot {(\frac{n}{2})}^{2}$ - 1 = $\frac{n^{2}}{2} - 1$
		
		Induction step: 
		\begin{align*}
		T(n) &= 4 \cdot T(\frac{n}{2})+ 3 \\
		&= 4 \cdot (\frac{n^{2}}{2} - 1) + 3 && \text{(Substituting hypothesis)}\\
		&= 4 \cdot (\frac{n^{2}}{2})- 4 + 3\\
		&= 2 \cdot n^{2} - 1 \\
		\text{Which is same as our hypothesis.} \\
		\therefore T(n) = 2 \cdot n^{2} -1.
		\end{align*}
	\end{enumerate}
	
\end{solution}

\end{document}